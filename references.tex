\bibliographystyle{abbrv}

\begin{thebibliography}{99}

\bibitem{polcyn1970} F.C. Polcyn, W.L. Brown, I.J. Sattinger, ``The Measurement of Water Depth by Remote Sensing Techniques'', Spacecraft Oceanography Project, US Naval Oceanographic Office, Washington DC, 1970

\bibitem{oneil1971} R. O'Neil, ``Algorithm AS 47: Function Minimization Using a Simplex Procedure'', \textit{Journal of the Royal Statistical Society. Series C (Applied Statistics)}, 
vol. 20, no. 3, pp. 338--345, 1971

\bibitem{morel1974} A. Morel, ``Optical properties of pure water and pure sea waters", \textit{Optical Aspects of Oceanography, N. G. Jerlov and E. S.Nielsen, eds. Academic}, pp. 1--24, 1974

% \bibitem{jerlov1976} N.G. Jerlov, ``Marine Optics'', \textit{Elsevier Oceanography Series}, New York, 1976

\bibitem{polycn1976} F.C. Polcyn, ``NASA/COUSTEAU OCEAN BATHYMETRY EXPERIMENT - Remote Bathymetry Using High Gain LANDSAT data'', \textit{NASA, Goddard Space Fligh Center}, 1976

\bibitem{jerlov1968} N. Jerlov, ``Optical Oceanography'', 1968

\bibitem{lyzenga1978} D.R. Lyzenga, ``Passive remote sensing techniques for mapping water depth and bottom features'', \textit{Applied Optics}, vol. 17, no. 3, pp. 379--383, 1978

%\bibitem{lyzenga1981} D.R. Lyzenga, ``Remote sensing of bottom reflectance and water attenuation parameters in shallow water using aircraft and LANDSAT data'', \textit{International Journal of Remote Sensing}, vol. 2, no. 1, pp. 71--82, 1981

\bibitem{nasa1985} \url{https://science.nasa.gov/missions/geosat}

\bibitem{gordon1988} H.R. Gordon et al, ``A semianalytical radiance model of ocean color", \textit{Journal of Geophysical Research}, vol. 93, pp. 10909--10942, 1988

\bibitem{maritorena1994} S. Maritorena, A. Morel, B. Gentili, ``Diffuse reflectance of oceanic shallow waters: Influence of water depth and bottom albedo'', \textit{American Society of Limnology and Oceanography}, vol. 39, no. 7, pp. 1689--1703, 1994

\bibitem{lee1994} Z. P. Lee, ``Visible-infrared remote-sensing model and applications for ocean waters", Ph.D. dissertation, Department of Marine Science, University of South Florida, 1994

\bibitem{lawler1995} A. Lawler, ``Sea-Floor Data Flow From Postwar Era'', \textit{Science}, vol. 270, issue 5237, pp. 727, November 1995

\bibitem{pope1997} R. Pope, E. Fry, ``Absorption spectrum  380--700 nm  of pure waters: II. Integrating cavity measurements", \textit{Applied Optics}, vol. 36, pp. 8710--8723, 1997

\bibitem{morel1998} Y. Morel, L.T. Lindell, ``PASSIVE MULTISPECTRAL BATHYMETRY MAPPING OF NEGRIL SHORES, JAMAICA'', 
\textit{Fifth International Conference on Remote Sensing for Marine and Coastal Environments}, San Diego, California, October 1998

\bibitem{lee1998} Z. Lee et al, ``Hyperspectral remote sensing
for shallow waters. I. A semianalytical model'', \textit{Applied Optics}, vol. 37, no. 27, pp. 6329--6338, 1998

\bibitem{lee1999} Z. Lee et al, ``Hyperspectral remote sensing for shallow waters: 2. Deriving bottom depths and water properties
by optimization'', \textit{Applied Optics}, vol. 38, no. 18, pp. 3831--3843, 1999

\bibitem{lee2001} Z. Lee et al, ``Properties of the water column and bottom derived from Airborne Visible Infrared Imaging Spectrometer (AVIRIS) data", \textit{Journal of Geophysical Research}, vol. 106, no. C6, pp. 11639--11651, June 2001

\bibitem{lee2002} Z. Lee, K.L. Carder, ``Effect of spectral band numbers on the retrieval of water column and bottom properties
from ocean color data'', \textit{Applied Optics}, vol. 41, no. 12, pp. 2191--2201, April 2002

%\bibitem{egbert2002} G.D. Egbert, S.Y. Erofeeva, ``Efficient inverse modelling of barotropic ocean tides'', \textit{Journal of Atmospheric and Oceanic Technology}, vol. 9, no. 2, pp. 183--204, 2002

\bibitem{albert2003} A. Albert, C.D. Mobley, ``An analytical model for subsurface irradiance and remote sensing reflectance in deep and shallow case-2 waters'', \textit{Optics Express}, vol. 11, no. 22, pp. 2873--2890, 2003

\bibitem{gege2004} P. Gege, ``The water color simulator WASI: an integrating software tool for analysis and simulation of optical in situ spectra'', \textit{Computers \& Geosciences}, vol. 30, pp. 523--532, 2004

\bibitem{klonowski2004} W.M. Klonowski, L.J. Majewski, P.R.C.S. Fearns, L.A. Clementson, M.J. Lynch, 2004. ``Bottom type classification using hyperspectral imagery'', \textit{Proceedings Ocean Optics XVII (CDROM)}, 2004

\bibitem{csiro2005} R. Babcock, L. Clementson, Bunbury Deployment 2003--2005: absorption-phytoplankton data, Integrated Marine Observing System (IMOS), \url{https://portal.aodn.org.au/}, CSIRO, 2003--2005

\bibitem{albert2006} A. Albert, P. Gege, ``Inversion of irradiance and remote sensing reflectance in shallow water between 400 and 800 nm for calculations of water and bottom properties'', \textit{APPLIED OPTICS}, vol. 45, no. 10, pp. 2331--2343, April 2006

\bibitem{mobley2005} C.D. Mobley et al, ``Interpretation of hyperspectral remote-sensing imagery by spectrum matching and look-up tables'', \textit{Applied Optics}, vol. 44, no. 17, pp. 3576--3592, 2005

\bibitem{lyzenga2006} D.R. Lyzenga, N.P. Malinas, F.J. Tanis, ``Multispectral Bathymetry Using a Simple Physically Based Algorithm'', \textit{IEEE Transactions on Geoscience and Remote Sensing}, vol. 44, no. 8, 2006

\bibitem{wettle2006} M. Wettle, V.E. Brando, ``SAMBUCA Semi-Analytical Model for Bathymetry, Un-mixing, and Concentration Assessment'', \textit{CSIRO Land and Water Science Report 22/06}, July 2006

%\bibitem{xu2006} H. Xu, ``Modification of normalised difference water index (NDWI) to enhance open water features in remotely sensed imagery", \textit{International Journal of Remote Sensing}, vo. 27, no. 14, pp. 3025--3033, July 2006

%\bibitem{6SV} E. Vermote, D. Tanre, J.L. Deuze, M. Herman, J.J. Morcrette, S.Y. Kotchenova, ``Second Simulation of a Satellite Signal in the Solar Spectrum - Vector (6SV)'', November 2006

\bibitem{klonowski2007} W.M. Klonowski, P.R.C.S. Fearns, M.J. Lynch, ``Retrieving key benthic cover types and bathymetry from hyperspectral imagery'', \textit{Journal of Applied Remote Sensing}, vol. 1, 011505, 2007

\bibitem{lee2009} Z. Lee, ``Test, Evaluate, and Characterize a Remote-Sensing Algorithm for Optically-Shallow Waters'', Geosystems Research Institute, Northern Gulf Institute, Mississippi State University, Stennis Space Center, 2009

\bibitem{hedley2009} J. Hedley, C. Roelfsema, S. Phinn, ``Efficient radiative transfer model inversion for remote sensing applications'', \textit{Remote Sensing of the Environment}, vol. 133, issue 11, pp. 2527--2532, November 2009

\bibitem{brando2009} V.E. Brando, J.M. Anstee, M. Wettle, A.G. Dekker, S.R. Phinn, C. Roelfsema, ``A physics based retrieval and quality assessment of bathymetry from suboptimal hyperspectral data'', \textit{Remote Sensing of the Environment}, vol. 113, issue 4, pp. 755--770, 2009

\bibitem{sagar2010} S. Sagar, M. Wettle, ``Mapping the fine-scale shallow water bathymetry of the Great Barrier Reef using ALOS AVNIR-2 data'', OCEANS 2010 IEEE - Sydney

\bibitem{lee2010} Z. Lee, ``Global Shallow-Water Bathymetry From Satellite Ocean Color Data'', \textit{EOS}, vol. 91, no. 46, 16 November 2010

\bibitem{ohlendorf2011} S. Ohlendorf, A. Muller, T. Heege, S. Cerdeira-Estrada, H.T. Kobryn, ``Bathymetry mapping and sea floor classification using multispectral satellite data and standardized physics-based data processing'', \textit{Remote Sensing of the Ocean, Sea Ice, Coastal Waters, and Large Water Regions}, Prague, September 2011

\bibitem{dekker2011} A. Dekker et al, ``Intercomparison of shallow water bathymetry, hydro-optics, and benthos mapping techniques in Australian and Caribbean coastal environments'', \textit{Limnology and Oceanography: Methods}, vol. 9, pp. 396--425, 2011

\bibitem{dekker2012} A.G. Dekker, S. Sagar, V.E. Brando, D. Hudson, ``Bathymetry from satellites for hydrographic purposes'', 
\textit{Shallow Survey 2012}, Wellington, New Zealand, February 2012

\bibitem{hedley2018} J.D. Hedley, et al, ``Coral reef applications of Sentinel-2: Coverage, characteristics, bathymetry and benthic mapping with comparison to Landsat 8'', \textit{Remote Sensing of Environment}, vol. 216, pp. 598--614. 

\bibitem{vanhellemont2019} Q. Vanhellemont, ``Adaptation of the dark spectrum fitting atmospheric correction for aquatic applications of the Landsat and Sentinel-2 archives'', \textit{Remote Sensing of Environment}, vol. 225, pp. 175--192, 2019. (\url{https://doi.org/10.1016/j.rse.2019.03.010})

\bibitem{open_street_maps} OpenStreetMap contributors, \url{https://www.openstreetmap.org/}

\end{thebibliography}
